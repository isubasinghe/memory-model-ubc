% the sample slide is created with 16:9 aspect ratio
\documentclass[aspectratio=169]{beamer}

% remove the options if you do not want to have them
\usetheme[
	% background=images/background.jpg, % you can add your own background image
	logo=images/unsw-portrait.png,
	sidelogo=images/unsw-landscape.png,
]{unsw}
% uncomment to show notes. Works very nicely with dspdfviewer. You get something similar to PPT's presenter view.
%\usepackage{pgfpages}
%\setbeameroption{show notes on second screen}

% information for the title page
\author{Isitha Subasinghe}
\title{Verification of programs in the presence of shared memory under weak memory consistency}
\subtitle{A memory model for shared memory systems}
\institute{School of Computer Science and Engineering}
\date{\today}

\begin{document}
	% use plain option to remove the page number from the title slide
	\begin{frame}[plain]
		\titlepage
	\end{frame}
	
	\begin{frame}{Introduction}
    \framesubtitle{The seL4 core platform and sDDF}  
    The seL4 core platform is a lightweight operating system intended to replace CAmkES. 
    This component was verified as of the last NCSC delivery phase. Building on top of this exists 
    the sDDF which provides a mechanism for writing device drivers. This component is unverified but the intention is to eventually. 
	\end{frame}

  \begin{frame}{Related Work}
    There have been some work on verifying low level systems, in fact the seL4 relies upon Tuch's memory model. 
    However, as far as I am aware, no model exists that takes into consideration various types of shared memory. 
  \end{frame}

  \begin{frame}{Tuch's model of memory}
    Tuch's \cite{tuch2007types} memory model considers programs.  
  \end{frame}

  \begin{frame}{Logical model of memory}
  \end{frame}
  
  \begin{frame}{Logical model of memory}
    \framesubtitle{Integer to pointer casts}
    Well unfortunately this isn't good enough for general C programs. It turns out those pesky C programmers do insane things such as 
    converting integers to pointers and back. There are naive solutions to tackle this problem, but unfortunately they come with serious downsides. 
    \begin{itemize}
      \item View all integers as pointers
      \item An analysis pass to determine which integer variables needed to be reasoned about as pointers. 
    \end{itemize}
  \end{frame}

  \begin{frame}{Concrete model of memory}
  \end{frame}

  \begin{frame}{Aliasing}

  \end{frame}

	\begin{frame}{Solutions}
		\framesubtitle{This is a subtitle}
		\begin{block}{Standard Block}
			This is a standard block.
		\end{block}
		
		\begin{exampleblock}{Example}
			This is an example block.
		\end{exampleblock}
		
    \begin{alertblock}{Integer to pointer casts}
			This is an alert block.
		\end{alertblock}
	\end{frame}

	
	\begin{frame}{Math}
		Mathematics is the queen of sciences and arithmetic is the queen of mathematics.

		\begin{align*}
			\vec{x} &= \vec{q}_{s, k} = \vec{R}_{s, k} = 
			\begin{bmatrix}
			\vec{R}_{s, x, k} & \vec{R}_{s, y, k} & \vec{R}_{s, z, k}
			\end{bmatrix} \\
			& \vec{p}_{lhip, k} = \vec{p}_{pelv, k} + d_{pelv}/2*\vec{R}_{pelv, y, k} \\
			& \vec{p}_{lkne, k} = \vec{p}_{lank, k} + d_{ltib}*\vec{R}_{ltib, z, k} \\
			& (\vec{p}_{lhip, k} - \vec{p}_{lkne, k} ) \cdot \vec{R}_{ltib, y, k} = 0\\
			& ||\vec{p}_{lhip, k} - \vec{p}_{lkne, k}||_2 = d_{lfem}
		\end{align*}	
		
		% these notes will only show when you uncomment 
		\note[item]{sample note 1}
		\note[item]{sample note 2}	
		
	\end{frame}

	\begin{frame}{Two Columns}
		We can also add two columns in the slides.
		\begin{columns}[t]
			\begin{column}[T]{0.4\textwidth}
				This is the first column. In this column, we can also add a block for instance.
				\vspace{1em}
				\begin{block}{Block}
					I am a block in a column.
				\end{block}
			\end{column}
			\begin{column}[T]{0.4\textwidth}
				\begin{itemize}
					\item In this column,
					\item we just add the
					\item bullet points.
				\end{itemize}
			\end{column}
		\end{columns}
	\end{frame}
	\begin{frame}{Acknowledgements}
		This theme is based on Kailash Budhathoki's \href{https://github.com/kailashbuki/beamerthemesaarland}{Saarland Beamer Theme}.
		
		Color guide was based from the visual guide found at the \href{https://www.brand.unsw.edu.au/download/}{UNSW brand hub}.
	\end{frame}

  \begin{frame}[allowframebreaks]{References}
    \bibliographystyle{amsalpha}
    \nocite{*}
    \bibliography{ref}
  \end{frame}
\end{document}
